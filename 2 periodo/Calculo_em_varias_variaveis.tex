\documentclass[12pt]{article}

\usepackage{setspace}
\usepackage{amssymb}
\usepackage{amsmath}
\usepackage{amsfonts}
\usepackage{wasysym}
\usepackage{graphicx}
\usepackage[pdftex,bookmarks=true,bookmarksopen=false,bookmarksnumbered=true,colorlinks=true,linkcolor=black]{hyperref}
\usepackage[utf8]{inputenc}
\usepackage{float}
\usepackage{pdfpages}


\usepackage[brazil]{babel}

\pagestyle{plain}

\newtheorem{theorem}{Teorema}[section]
\newtheorem{corollary}{Corolário}[theorem]
\newtheorem{lemma}[theorem]{Lema}
\newtheorem{definition}{Definição}

\begin{document}

\begin{titlepage}
\begin{center}
\textbf{\LARGE Fundação Getulio Vargas}\\ 
\textbf{\LARGE Escola de Matemática Aplicada}

\par
\vspace{170pt}
\textbf{\Large Wellington José}\\
\vspace{32pt}
\textbf{\Large Resumo de cálculo em várias variáveis}\\
\end{center}

\par
\vfill
\begin{center}
{{\normalsize Rio de Janeiro}\\
{\normalsize \the\year}}
\end{center}
\end{titlepage}

\section*{Capítulo 16 - Cálculo Vétorial}
\subsection*{16.1 - Campos Vetoriais}
\begin{definition}
Seja $E$ um conjunto em $\mathbf{R}^n$. Um campo vetorial em $\mathbf{R}^n$ é uma função $F$ que associa a cada ponto $(x_1, x_2, \dots, x_n) \in E$ um vetor $F(x_1, x_2, \dots, x_n)$.
\end{definition}

\begin{definition}[Campo vetorial gradiente]
Se $f: E \in \mathbf{R}^n \rightarrow{} \mathbf{R}^n$, o campo vetorial gradiente é dado por:

$$\nabla f(x_1, x_2, \dots, x_n) = f_{x_1} (x_1, x_2, \dots, x_n)j_1 + \dots + f_{x_n} (x_1, x_2, \dots, x_n)j_n \ \ \footnote{Note que, $f_x$ é a derivada de f em relação a x}$$
\end{definition}

\subsection*{16.2 - Integrais de Linha}
\begin{definition}[integral de linha sobre curva]
Se $f$ é definida sobre uma curva suave $C$ dada por uma equação paramétrica da forma $x = x(t), y = y(t)$ com $a \leq t \leq b$. Então a integral de linha de $f$ sobre $C$ é:

$$\int_{C} f(x, y) ds = \lim_{n \rightarrow{} \infty} \sum_{j=1} ^n f({x_i}^*, {y_i}^*) \Delta s_i$$

ou usando a seção 10.2

$$\int_{C} f(x, y) ds = \int_{a}^{b} f(x(t), y(t)) \sqrt{\left(\frac{dx}{dt}\right)^2 + \left(\frac{dy}{dt}\right)^2} dt$$
\end{definition}

\subsubsection*{Integral de linha com relação ao comprimento do arco}
Podemos escrever integral de linha em função de $t: x = x(t), y = y(t), dx = x'(t)dt, dy = y'(t)dt$ ficando com:

$$\int_{C} f(x, y) dx = \int_{a}^{b} f(x(t), y(t))x'(t)dt$$
$$\int_{C} f(x, y) dy = \int_{a}^{b} f(x(t), y(t))y'(t)dt$$

\subsubsection*{Integrais de Linha no Espaço}
De forma análoga a integrais duplas
$$\int_{C} f(x, y, z) ds = \int_{a}^{b} f(x(t), y(t), z(t)) \sqrt{\left(\frac{dx}{dt}\right)^2 + \left(\frac{dy}{dt}\right)^2 + \left(\frac{dz}{dt}\right)^2} dt$$

e

$$\int_{C} f(x, y, z) dz = \int_{a}^{b} f(x(t), y(t), z(t))z'(t)dt$$

\subsubsection*{Integrais de Linha de Campos Vetorias}

Seja $F$ um campo vetorial contínuo definido sobre uma curva suave $C$ dada pela função vetorial $r(t), a \leq t \leq b$. Então, a integral de linha de $F$ ao longo de $C$ é

$$\int_C F dr = \int_{a}^{b} F(r(t)).r'(t) dt = \int_C F . T ds$$
onde $T(x, y, z)$ é o vetor tangente unitário no ponto $(x, y, z) \in C$.

\subsection*{16.3 - O Teorema Fundamental das Integrais de Linha}
\begin{theorem}
    Seja C uma curva suave dada pela função vetorial $r(t), a \leq t \leq b$. Seja $f$ uma função diferenciável de duas ou três variáveis cujo vetor gradiente $\nabla f$ é contínuo em C. Então

    $$\int_C \nabla f . dr = f(r(b)) - f(r(a))$$
\end{theorem}

obs.: Lembre-se que, $\nabla f = \textbf{F}$

\subsubsection*{Independência do Caminho}

Suponha que $C_1$ e $C_2$ sejam curvas suaves por partes que tem mesmos pontos iniciais e finais A e B, se $\nabla f$ é contínua então

$$\int_{C_1} \nabla f . dr = \int_{C_2} \nabla f . dr$$

\begin{theorem}
    $\int_C F.dr$ é independente do caminho em D se e somente se $\int_C F . dr = 0$ para todo caminho fechado $C \in D$.
\end{theorem}

\begin{theorem}
    Suponha que F seja um campo vetorial contínuo em uma região aberta conexa por caminhos D. Se $\int_C F . dr$ for independente do caminho em D, então F é um campo vetorial conservativo em D, ou seja, existe uma função $f$ tal que $\nabla f = F$.
\end{theorem}

\begin{theorem}
    Se $F(x, y) = P(x, y)i + Q(x, y)j$ é um campo vetorial conservativo, onde P e Q têm derivadas parciais de primeira ordem contínuas em um domínio D, então em todos os pontos de D temos

    $$\frac{\partial P}{\partial y} = \frac{\partial Q}{\partial x}$$
\end{theorem}

\begin{theorem}
    Seja $F = Pi + Qj$ um campo vetorial em uma região aberta simplesmente conexa D. Suponha que P e Q tenham derivadas contínuas de primeira ordem e que

    $$\frac{\partial P}{\partial y} = \frac{\partial Q}{\partial x}$$

    Em todo D, então F é conservativo.
\end{theorem}

\subsection*{16.4 - Teorema de Green}
O teorema de Green fornece a relação entre uma integral de linha ao redor de uma curva fechada simples C e uma integral sobre a região do plano D delimitada de C.

\begin{theorem}[Teorema de Green]
    Seja C uma curva plana simples, fechada, contínua por partes, orientada positivamente, e seja D a região delimitada por C. Se $F_1$ e $F_2$ têm derivadas parciais de primeira ordem contínuas sobre uma região aberta que contenha D, então
    
    $$\int_{C} F_1 d x + F_2 d y = \iint_D \left( \dfrac{\partial F_2}{\partial x} - \dfrac{\partial F_1}{\partial y} \right) d A$$
\end{theorem}

\begin{corollary}
    Do Teorema de Green podemos tirar a área de D
    
    $$A = \dfrac{1}{2} \oint_C x d y - y d x$$
\end{corollary}

obs.: $\oint_C F_1 d x + F_2 d y$ indica que a integral de linha é calculada usando a orientação positiva da curva fechada C.

\subsection*{16.5 - Rotacional e Divergente}
\subsubsection*{Rotacional}
Se $F = F_1 \textbf{i} + F_2 \textbf{j} + F_3 \textbf{k}$ é um campo vetorial em $\mathbb{R}^3$ e as derivadas parciais de $F_1, F_2$ e $F_3$ existem, então o \textbf{rotacional} de \textbf{F} é o campo vetorial em $\mathbb{R}^3$ definido por

$$\text{rot } \textbf{F} = \left( \dfrac{\partial F_3}{\partial y} - \dfrac{\partial F_2}{\partial z} \right) \textbf{i} + \left( \dfrac{\partial F_1}{\partial z} - \dfrac{\partial F_3}{\partial x} \right) \textbf{j} + \left( \dfrac{\partial F_2}{\partial x} - \dfrac{\partial F_1}{\partial y} \right) \textbf{k}$$

Ou,

$$\text{rot } \textbf{F} = \left| \begin{array}{ccc}
    i & j & k \\
    \frac{\partial}{\partial x} & \frac{\partial}{\partial y} & \frac{\partial}{\partial z} \\
    F_1 & F_2 & F_3
\end{array} \right|$$

\begin{theorem}
    Se $f$ é uma função de três variáveis e tem derivadas parciais de segunda ordem contínuas, então

    $$\text{rot} (\nabla f) = 0$$
\end{theorem}

\begin{theorem}
    Se \textbf{F} for um campo vetorial definido sobre todo $\mathbb{R}^3$ cujas funções $F_1, F_2$ e $F_3$ possuem derivadas de segunda ordem contínuas e rot \textbf{F} $ = 0$, então \textbf{F} será um campo vetorial consecutivo.
\end{theorem}

\subsubsection*{Divergente}
Se $F = F_1 i + F_2 j + F_3 k$ é um campo vetorial em $\mathbb{R}^3$ e $F_1, F_2, F_3$ possuem derivadas, então o \textbf{divergente} de F é

$$\text{div } F = \dfrac{\partial F_1}{\partial x} + \dfrac{\partial F_2}{\partial y} + \dfrac{\partial F_3}{\partial z}$$

\begin{theorem}
    Se $F = F_1 i + F_2 j + F_3 k$ é campo vetorial sobre $\mathbb{R}^3$ e $F_1, F_2, F_3$ têm derivadas parciais de segunda ordem contínuas, então

    $$\text{div rot } F = 0$$
\end{theorem}

\subsection*{16.6 - Superfícies Parametrizadas e suas Áreas}

Podemos descrever uma superfície por uma função vetorial de dois parâmetros u e v, em vez de apenas um único t.

$$r(u, v) = x(u, v)i + y(u, v)j + z(u, v)k$$

\subsubsection*{Superfícies de Revolução}

Uma superfície de revolução num certo eixo x, com uma função $f$:

$$x = x, y = f(x) \cos{\theta}, z = f(x) \sin{\theta}$$

\subsubsection*{Planos Tangentes}

O plano tangente de uma certa função vetorial $r(u, v) = x(u, v)i + y(u, v)j + z(u, v)k$, no ponto $P_0$ com vetor posição $r(u_0, v_0)$ é dada por:

$$r_v = \dfrac{\partial x}{\partial v}(u_0, v_0)i + \dfrac{\partial y}{\partial v}(u_0, v_0)j + \dfrac{\partial z}{\partial v}(u_0, v_0)k$$

$$r_u = \dfrac{\partial x}{\partial u}(u_0, v_0)i + \dfrac{\partial y}{\partial u}(u_0, v_0)j + \dfrac{\partial z}{\partial u}(u_0, v_0)k$$

Onde o vetor normal do plano tangente é dado por $|r_v \times r_u| = \alpha i + \beta j + \gamma k$, note que $\alpha, \beta, \gamma$ estão em função de $u, v$.

E a equação do plano num ponto $(x_0, y_0, z_0)$:

$$\alpha(u_0, v_0)(x - x_0) + \beta(u_0, v_0)(y - y_0) + \gamma(u_0, v_0)(z - z_0) = 0$$

\subsubsection*{Área da Superfície}
\subsubsection*{Definição}
Se uma superfície parametrizada suave S é dada pela equação $r(u, v) = x(u, v)i + y(u, v)j + z(u, v)k$, com $(u, v) \in D$ e S é coberta uma única vez quando $(u, v)$ abrange todo o domínio D parâmetros, então a área da superfície de S é:

$$A(S) = \iint_D |r_u \times r_v| dA$$

onde 

$$r_u = \dfrac{\partial x}{\partial u}(u_0, v_0)i + \dfrac{\partial y}{\partial u}(u_0, v_0)j + \dfrac{\partial z}{\partial u}(u_0, v_0)k$$ 

$$r_v = \dfrac{\partial x}{\partial v}(u_0, v_0)i + \dfrac{\partial y}{\partial v}(u_0, v_0)j + \dfrac{\partial z}{\partial v}(u_0, v_0)k$$

\subsubsection*{Exemplo - Determine a área da esfera de raio a}

Como foi visto no capítulo 15 temos as equações paramétricas:

$$ x = a \sin{\phi} \cos{\theta}, y = a \sin{\phi} \sin{\theta}, z = a \cos{\phi}$$

O produto cruzado dos vetores tangentes:

$$r_\phi \times r_\theta = \left| \begin{array}{rcr}
i & j  & k \\\\
 \dfrac{\partial x}{\partial \phi} & \dfrac{\partial y}{\partial \phi} & \dfrac{\partial z}{\partial \phi}\\\\
 \dfrac{\partial x}{\partial \theta} & \dfrac{\partial y}{\partial \theta}  & \dfrac{\partial z}{\partial \theta}
\end{array} \right| = \left| \begin{array}{rcr}
i & j  & k \\ 
 a \cos{\phi} \sin{\theta} & a \sin{\phi} \cos{\theta} & -a \sin{\phi}\\
 -a \sin{\phi} \sin{\theta} & a \sin{\phi} \cos{\theta}  & 0
\end{array} \right|$$

$$ = a^2 \sin{\phi}^2 \cos{\theta}i + a^2 \sin{\phi}^2 \sin{\theta}j + a^2 \sin{\phi} \cos{\phi}k$$

Logo,

$$|r_\phi \times r_\theta| = \sqrt{a^4 \sin{\phi}^4 \cos{\theta}^2 + a^4 \sin{\phi}^4 \sin{\theta}^2 + a^4 \sin{\phi}^2 \cos{\phi}^2} = a^2 \sin{\phi}$$

Uma vez que $\sin{\phi} \geq 0$ para $0 \leq \phi \leq \pi$. Portanto, a área da esfera é

$$A = \iint_D |r_u \times r_v| dA = \int_0 ^{2 \pi} \int_0 ^\pi a^2 \sin{\phi} d \phi d \theta = 4 \pi a^2$$

\subsubsection*{Área de Superfície do Gráfico de uma Função}

Para o caso especial de uma superfície S com equação $z = f(x, y)$, onde $(x, y) \in D$ e $f$
tem derivadas parciais contínuas, tomamos x e y como parâmetros. As equações paramétricas são

$$x = x, y = y, z = f(x, y)$$

Então a área fica

$$A(S) = \iint_D \sqrt{1 + \left(\frac{\partial z}{\partial x}\right)^2 + \left(\frac{\partial z}{\partial y}\right)^2} dA$$

\subsection*{16.7 - Integrais de Superfície}
\subsubsection*{Superfícies parametrizadas}

Dada uma superfície $r(u, v) = x(u, v)i + y(u, v)j + z(u, v)k$ com $(u, v) \in D$, a superfície é dada por:

$$\iint_S f(x, y, z) dS = \iint_D f(r(u, v)) |r_u \times r_v| dA$$

\subsubsection*{Gráficos}
Como caso particular se $z = f(x, y)$ podemos calcular a superfície com equações parametrizadas

$$x = x, y = y, z = f(x, y)$$

E a superfície fica:

$$\iint_S f(x, y, z) dS = \iint_D f(x, y, f(x, y)) \sqrt{\left(\frac{\partial z}{\partial x}\right)^2 + \left(\frac{\partial z}{\partial y}\right)^2 + 1} dA$$

\subsubsection*{Superfícies Orientadas}
Dada uma superfície $z = g(x, y)$ orientada, onde sua orientação é dada pelo vetor unitário

$$n = \dfrac{-\dfrac{\partial g}{\partial x}i - \dfrac{\partial g}{\partial y}j + k}{\sqrt{1 + \left(\dfrac{\partial g}{\partial x}\right)^2 + \left(\dfrac{\partial g}{\partial y}\right)^2}}$$

E se S for uma superfície orientada suave com parametrização vetorial $r(u, v)$, então pode ser associada à orientação do vetor normal unitário.

$$n = \dfrac{r_u \times r_v}{|r_u \times r_v|}$$

\subsubsection*{Integrais de Superfície de Campos Vetoriais}

Se F for um campo vetorial contínuo definido sobre uma superfície orientada S com vetor normal unitário n, então a superfície integral de F sobre S é

$$\iint_S F d S = \iint_S F \cdot n d S$$

obs.: A integral de superfície de um campo vetorial sobre S é igual à superfície de sua componente normal em S.

Se S é uma função vetorial dada por $r(u, v)$, então tomando D o domínio dos parâmetros:

$$\iint_S F dS = \iint_D F \cdot (r_u \times r_v) d A$$

\subsection*{16.8 - Teorema de Stokes}
Sejam S uma superfície orientável de classe $C^1$, com bordo e orientado coerentemente e $\Vec{F}$ um campo vetorial de classe $C^1$ definido em um domínio que contém S. Então

$$\int_{\partial S} F d r = \iint_S \text{rot} F \cdot d S =  \iint_S \text{rot} F \cdot n d A$$

\subsection*{16.9 - Teorema do Divergente (ou Teorema de Gauss)}
O Teorema do Divergente é uma generalização do Teorema de Green (16.5) para o espaço com certas hipóteses, para que vala 

$$\iint_S F \cdot n d S = \iiint_E \ \text{div} \ F(x, y, z) d V$$

Onde S é a superfície fronteira da região sólida E.

\begin{theorem}[O Teorema do Divergente]
    Seja E uma região sólida simples e seja S a superfície fronteira de E, orientada positivamente (para fora). Seja F um espaço vetorial cujas funções componentes tenham derivadas parciais contínuas em uma região aberta que contenha E. Então
    
    $$\iint_S F \cdot d S = \iiint_E \ \text{div} \ F d V$$
\end{theorem}

\subsection*{16.10 - Resumo resumido do resumo}

\begin{theorem}[Teorema Fundamental do Cálculo]
    $$\int_a^b F'(x) d x = F(b) - F(a)$$
\end{theorem}

\begin{theorem}[Teorema Fundamental para as Integrais de Linha]
    $$\int_C \nabla f \cdot d r = f(r(b)) - f(r(a))$$
\end{theorem}

\begin{theorem}[Teorema de Green]
    $$\iint_D \left( \dfrac{\partial Q}{\partial x} - \dfrac{\partial P}{\partial y} \right) \ d A = \int_C P d x + Q d y$$
\end{theorem},

\begin{theorem}[Teorema de Stokes]
    $$\iint_S \text{ rot } F \cdot d S = \int_C F \cdot d r$$
\end{theorem}

\begin{theorem}[Teorema de Gauss]
    $$\iiint_E \text{ div } F \ d V = \iint_S F \cdot d S$$
\end{theorem}

\end{document}
